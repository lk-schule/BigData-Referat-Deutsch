\documentclass[10pt,a4paper]{beamer}

\include{beamer-with-code}

\author{Luca Hülsmann \and Luca Kiebel \and Luca Hartmann}
\title{Big Data}
\date{\today}
\setlength{\itemsep}{10pt}
\begin{document}
\begin{frame}
\titlepage
\end{frame}

\begin{frame}
\frametitle{Inhaltsverzeichnis}\tableofcontents
\end{frame}

\section{Was ist Big Data?}
\subsection{Definition}
\begin{frame}
\frametitle{Definition}
\framesubtitle{Was ist Big Data?}
Big Data bezeichnet das Sammeln und Auswerten großer Datenmengen aus unterschiedlichen Quellen in Echtzeit und wird somit oft mit Cloud Computing in Verbindung gebracht.
\pftn{http://www.searchenterprisesoftware.de/definition/Big-Data}
\end{frame}
\begin{frame}
\frametitle{Definition}
\framesubtitle{Was ist Big Data?}
Einige Quellen sind:
\begin{itemize}
	\item Aufzeichnungen von unterschiedlichen Überwachungssystemen \pause
	\item jegliche elektronische Kommunikation \pause
	\item Social Media \pause
	\item Nutzung von Kunden- beziehungsweise Bankkarten
\end{itemize}
\pftn{https://de.wikipedia.org/wiki/Big\_Data}
\end{frame}

\subsection{Wofür wird Big Data genutzt?}
\begin{frame}
\frametitle{Wofür wird Big Data genutzt?}
\framesubtitle{Was ist Big Data?}
\begin{itemize}
\item Verhaltensanalyse
	\begin{itemize}
	\item Erkennen was wann gekauft wird
	\item Erkennen wieso etwas gekauft wird
	\end{itemize} \pause
\item Targeted Advertising
	\begin{itemize}
	\item Personalisierte Werbung
	\item Coupons anbieten 
	\end{itemize} \pause
\item Predictive Support
	\begin{itemize}
	\item Erkennen von Potenziellen Schwachstellen
	\item Austauschen von Teilen bevor Abnutzung bemerkbar wird
	\end{itemize}
\pftn{https://www.slideshare.net/Dell/big-data-use-cases-36019892}
\end{itemize}
\end{frame}

\subsection{Wer nutzt Big Data?}
\begin{frame}
\frametitle{Wer nutzt Big Data?}
\framesubtitle{Was ist Big Data?}
\begin{itemize}
\item \( 1/3\) deutscher Unternehmen \pause
\item 2014: 23\% \pause
\item Hilfe zu relevaten Entscheidungen bei 80\% \pause
\item 2020: 1,7 MB pro Person in einer Sekunde \ftn{1}{https://blog.capterra.com/10-surprising-big-data-statistics/}
\pftn{https://www.bitkom.org/Presse/Presseinformation/Jedes-dritte-Unternehmen-nutzt-Big-Data.html}
\end{itemize}
\end{frame}

\section{Welche Vorteile haben Unternehmen durch Big Data?}
\begin{frame}
\frametitle{Welche Vorteile haben Unternehmen durch Big Data?}
\begin{itemize}
	\item Big Data bringt Verbesserungen in allen Funktionsbereichen eines Unternehmens \pause
	\item Wird im allgemeinen für die Umsetzung von Unternehmenszielen genutzt \pause
	\item Strategisches Kunden und Forschungs Management
\end{itemize}
\pftn{https://bigdatablog.de/2014/12/22/ueber-nutzen-und-moeglichkeiten-von-big-data/}
\end{frame}

\section{Welche ethischen Probleme kommen durch Big Data auf?}
\begin{frame}
\begin{itemize}
  \frametitle{Welche ethischen Probleme kommen durch Big Data auf?}		  
\frametitle{Welche ethischen Probleme kommen durch Big Data auf?}
\item keine Möglichkeit sich auf Immaterialgüterrechte zu beziehen (Patent, Urheberrecht, Markenrecht, Designschutz)\pause
\item Datenbankschutzrecht gilt auch nicht (schützt nur Investionen in Datenbanken, nicht die Datenbanken an sich)
\pftn{https://www.computerwoche.de/a/wem-gehoeren-die-daten-im-internet-of-things,3328337}
\end{itemize}
\end{frame}

\section{Bewertung}
\subsection{Vorteile}
\begin{frame}{Vorteile}{Bewertung}
\begin{itemize}
\item Optimierung von Geschäftsprozessen \pause
\item Kalkulieren von Risiken \pause
\item Werbung an Kunden ausrichten
\end{itemize}
\end{frame}

\subsection{Nachteile}
\begin{frame}{Nachteile}{Bewertung}
\begin{itemize}
\item Gläserner Mensch
\end{itemize}
\end{frame}

\subsection{Fazit}
\begin{frame}{Fazit}{Bewertung}
\begin{itemize}
\item Nutzen überwiegt nicht Probleme \pause
\item Gesetze für Umgang mit Daten \pause
\item Voteile könnten ohne Nachteile bleiben
\end{itemize}
\end{frame}

\end{document}

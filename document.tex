\documentclass[10pt,a4paper]{beamer}

\usepackage[ngerman]{babel}
\usepackage[T1]{fontenc}
\usepackage[utf8]{inputenc}
\usepackage{varioref}
\usepackage{hyperref}
\usepackage{cleveref}
\usepackage{amsmath}
\usepackage{amsfonts}
\usepackage{amssymb}
\usepackage{makeidx}
\usepackage{graphicx}
\usepackage{csquotes}
\usepackage[most]{tcolorbox}
\usepackage{amssymb}
\usepackage{lmodern}

\setbeamercolor{logo}{bg=white}  %controls the color of the logo area
\setbeamertemplate{navigation symbols}{}
\usetheme{PaloAlto}
\renewcommand{\footnotesize}{\small}
\newcommand{\pftn}[1]{\let\thefootnote\relax\footnotetext{\tiny #1}}
\newcommand{\ftn}[2]{\footnote[#1]{\tiny #2}}

\newenvironment{hlbox}{\begin{tcolorbox}[enhanced,colback=white,colframe=white,sharpish corners,fuzzy halo=0.5mm with lightgray]}{\end{tcolorbox}}

\logo{\includegraphics[width=0.155\linewidth]{hbbk-logo}}
\author{Luca Hülsmann \and Luca Kiebel \and Luca Hartmann}
\title{Big Data}
\date{\today}
\setlength{\itemsep}{10pt}
\begin{document}
\begin{frame}
\titlepage
\end{frame}

\begin{frame}
\frametitle{Inhaltsverzeichnis}\tableofcontents
\end{frame}

\section{Was ist Big Data?}

\subsection{Definition}
\begin{frame}
\frametitle{Definition}
Ich bin cool
\end{frame}

\subsection{Wofür wird Big Data genutzt?}
\begin{frame}
\frametitle{Wofür wird Big Data genutzt?}
Wofür nicht?
\end{frame}

\subsection{Wer nutzt Big Data?}
\begin{frame}
\frametitle{Wer nutzt Big Data?}
Wer nicht?
\end{frame}

\section{Welche Vorteile haben Unternehmen durch Big Data?}
\begin{frame}
Inhalt...
\end{frame}

\section{Welche ethischen Probleme kommen durch Big Data auf?}
\begin{frame}
Inhalt...
\end{frame}

\section{Bewertung}
\begin{frame}
Inhalt...
\end{frame}

\end{document}
